%------------------------
% Resume Template
% Author : Anubhav Singh
% Github : https://github.com/xprilion
% License : MIT
%------------------------

%\documentclass[a4paper,30pt]{article}
\documentclass[10pt]{extreport}

\usepackage{latexsym}
\usepackage[empty]{fullpage}
\usepackage{titlesec}
\usepackage{marvosym}
\usepackage[usenames,dvipsnames]{color}
\usepackage{verbatim}
\usepackage{enumitem}
\usepackage[pdftex]{hyperref}
\usepackage{fancyhdr}

\pagestyle{fancy}
\fancyhf{} % clear all header and footer fields
\fancyfoot{}
\renewcommand{\headrulewidth}{0pt}
\renewcommand{\footrulewidth}{0pt}

% Adjust margins
\addtolength{\oddsidemargin}{-0.530in}
\addtolength{\evensidemargin}{-0.375in}
\addtolength{\textwidth}{1in}
\addtolength{\topmargin}{-.45in}
\addtolength{\textheight}{1in}

\urlstyle{rm}

\raggedbottom
\raggedright
\setlength{\tabcolsep}{0in}

% Sections formatting
\titleformat{\section}{
  \vspace{-10pt}\scshape\raggedright\large
}{}{0em}{}[\color{black}\titlerule \vspace{-6pt}]

%-------------------------
% Custom commands
\newcommand{\resumeItem}[2]{
  \item\small{
    \textbf{#1}{: #2 \vspace{-2pt}}
  }
}

\newcommand{\resumeItemWithoutTitle}[1]{
  \item\small{
    {\vspace{-2pt}}
  }
}

\newcommand{\resumeSubheading}[4]{
  \vspace{-1pt}
    \begin{tabular*}{1.0\textwidth}{l@{\extracolsep{\fill}}r}
      \textbf{#1} & \textbf{#2}  \vspace{1mm} \\
      {#3} & \textbf{#4} \\
    \end{tabular*}\vspace{-3pt}
}


\newcommand{\resumeSubItem}[2]{\resumeItem{#1}{#2}\vspace{-3pt}}

\renewcommand{\labelitemii}{$\circ$}

\newcommand{\resumeSubHeadingListStart}{\begin{itemize}[leftmargin=*]}
\newcommand{\resumeSubHeadingListEnd}{\end{itemize}}
\newcommand{\resumeItemListStart}{\begin{itemize}}
\newcommand{\resumeItemListEnd}{\end{itemize}\vspace{-5pt}}

%-----------------------------
%%%%%%  CV STARTS HERE  %%%%%%

\begin{document}

%----------HEADING-----------------
\vspace*{-40pt}
\begin{center}
	\textbf{{\LARGE Jack Hong}} \\
	      \vspace{2mm}

    (651) 248-2041 $|$ jack.h18870@gmail.com $|$ U.S. Citizen
\end{center}
\vspace{-4mm}

%-----------Skills-----------------
% \section{Skills}
% \vspace{-1.5mm}
%-----------Experience-----------------
\section{Work Experience}
\resumeSubheading{Flexport}{Feb 2023 - Feb 2024}
    {\textit{Software Engineer - Java, GraphQL, gRPC, Ruby, Typescript, React, Next.js, Postgres, AWS}}{}
    \vspace{-2mm}
    \begin{itemize}
    \item[\textperiodcentered] Reduced the multi-step label printing process to one click by creating a full-stack flow from scratch, requiring new Java Spring Boot backend service and React user interface.
        \vspace{-2mm}

    \item[\textperiodcentered] Increased the success rate of email scrapers from 40\% to 90\% by redesigning email scraping. This was a user driven process - operators in the Amsterdam warehouse brought up flakiness issues with prior scrapers. I used datadog to monitor scrapers and introduced new scrapers to track invoicing emails. This work reduced cost leakages in invoicing and enhancing operational efficiency by reducing tedious email overhead.
            \vspace{-2mm}

   \item[\textperiodcentered] Collaborated with cross functional teams to support Flexport’s integration with TipTop, a supply chain management tool. TipTop would send shipment events to Flexport, and another team would do some basic parsing of those events. I updated the Ruby backend to read these events and update warehouse relevant milestones. This improved operational efficiency by reducing the burden on operators.
           \vspace{-2mm}

   \item[\textperiodcentered] Collaborated with cross functional teams to support Flexport's integration with TipTop, a supply chain management tool. 
    \end{itemize}

\resumeSubheading{Meta}{May 2022 - Aug 2022}
    {\textit{Software Engineering Intern - Python, Postgres}}{}
    \vspace{-2mm}
    \begin{itemize}
     \item[\textperiodcentered] Introduced monitoring for Ads Machine Learning pipelines by using Python to collect status metrics.

     \vspace{-2mm}
     \item[\textperiodcentered] Improved time to resolution for oncall engineers by surfacing debug information to oncall engineers through Unidash (an internal dashboard tool). This information was sourced from the status metrics I collected.

     \vspace{-2mm}
     \item[\textperiodcentered] Used Python multi-threading to optimize metrics collection and updates.
    \end{itemize}
    %-----------Item 1---------------
    
    \resumeSubheading{Capital One}{Jun 2021 - Aug 2021}
    {\textit{Software Engineering Intern - Node.js}}{}
    \vspace{-2mm}
    \begin{itemize}
%  Worked on an all intern team. Hackathon-esque project in the Capital One banking app. 
% Presented project to Retail Direct Business Unit (Over 300 people)
    \item[\textperiodcentered] Worked on an all intern team to create a ATM Locator feature in the Capital One banking app.
            \vspace{-2mm}

        \item[\textperiodcentered] I created the node.js backend to support the ATM locator service.
                \vspace{-2mm}

        \item[\textperiodcentered] Our intern team presented our project to the Retail Direct business unit to over 300 stakeholders.
    \end{itemize}
    %-----------Item 1---------------
    
    \resumeSubheading{Intel}{Nov 2020 - May 2021}
    {\textit{DevOps Engineering Intern - Python, C}}{}
    \vspace{-2mm}
    \begin{itemize}
    % Basically hooked up Pytest with GHS for testing.
    \item[\textperiodcentered] Worked on Python testing infrastructure for FPGA and ASIC SSD controller using pytest. This allowed the CI/CD pipeline to run tests on real silicon - FPGAs and ASICs
            \vspace{-2mm}

    \item[\textperiodcentered] Wrote Python code to integrate Green Hills Software Debugger with the CI/CD pipelines to prevent regressions in SSD controller firmware. 
            \vspace{-2mm}

    \item[\textperiodcentered] My work 
 improved code quality across the Optane group - prior to my work code regressions were much harder to track and exceedingly common, due to the lack of device testing on deploys.
    \end{itemize}
    %-----------Item 2---------------
    
    \resumeSubheading{University of Michigan}{Sep 2019 - Dec 2022}
    {\textit{Lead Teaching Assistant}}{}
        \vspace{-2mm}
    \begin{itemize}
    \item[\textperiodcentered] Created the brand new EECS 370 website \& website infrastructure using GitHub pages and actions, now maintained by current staff.
        \vspace{-2mm}
    \item[\textperiodcentered] Created a brand new Branch Predictor based project for students to code in C. This required a new project spec, starter files, and an autograder.
        \vspace{-2mm}
    \item[\textperiodcentered] Oversaw over 800 students each semester in EECS 370, Introduction to Computer Organization
        \vspace{-2mm}
    \item[\textperiodcentered] Led office hours where students learned
    about C, C++, ARM assembly, virtual memory, caching, and processor pipelines
    \end{itemize}

    % \section{Skills}
    %     \underline{Programming Languages \& Tools}: Java, Java Spring, Javascript, Typescript, C, C++, Python, Ruby
    %   \vspace{2mm}


\vspace{-1.5mm}

    %-----------EDUCATION-----------------
\section{Education}
        \resumeSubheading{\underline{University of Michigan, Ann Arbor}} {Jan 2022 - Dec 2022}
      {Master of Science \& Engineering, Computer Science}{ GPA: 4.00}
     
      \vspace{2mm}
    \resumeSubheading{\underline{University of Michigan, Ann Arbor}}{Sept 2018 - Dec 2021}
      {Bachelor of Science \& Engineering, Computer Science}{GPA: 3.82}
            \vspace{2mm}

      {\textbf{Selected Courses:} Natural Language Processing (Python), Advanced Compilers (C++, LLVM), Distributed Systems (Go), Parallel Computer Architecture (Verilog), Parallel Programming with GPUs (C++, CUDA) , Computer Architecture (Verilog), Computer Vision (Python), Operating Systems (C++), Embedded Systems (C++, ARM), Technology to Optimize Human Learning (Python) }{}

\vspace{-1.5mm}
\vspace{-5pt}

\end{document}
